
\documentclass[12pt]{article}
\renewcommand{\familydefault}{\sfdefault}
\newenvironment{ppl}{\ttfamily}{\par}
\usepackage{graphicx}
\usepackage{hyperref}
\usepackage{lipsum}
\usepackage{amsmath}
%\usepackage{apacite}
\usepackage[round]{natbib}

\begin{document}

%\def\mymacro#1{{\it #1}}

%\title{December 25, 2017}

\section*{\large December 25, 2017}
\vspace{-0.1 in}

I just spent five hours of my Christmas Monday on installing {\bf gfortran} on MacOS High Sierra! I was getting the following error while running some simple Fortran codes for earthquake source simulations. I correctly thought it is due to some configuration problems of the latest version of gfortran --that I installed from the \href{https://gcc.gnu.org/wiki/GFortranBinariesMacOS}{GNU project}-- with recent MacOS. I was correct, it turned out that the latest releases (ver>7) of gfortran (maybe only for this particular use) is not compatible with High Sierra anymore. 

\begin{ppl}
FATAL:/opt/local/bin/../libexec/as/x86-64/as: I don't understand 'm' flag! 
\end{ppl}

A solution that I found after hours of struggle, was to install an older version {\bf gcc6.4} that with itself has {\bf gfortran6}. This could be installed via {\bf MacPorts}:

\begin{ppl}
sudo port install gcc6
\end{ppl}

and will have the following name in binaries folder: ``gfortran-mp-6''.


%




%\bibliographystyle{plainnat}
%\small
%\bibliography{library_segregation}
\end{document} 
  