
\documentclass[12pt]{article}
\renewcommand{\familydefault}{\sfdefault}
\newenvironment{ppl}{\ttfamily}{\par}
\usepackage{graphicx}
\usepackage{hyperref}
\usepackage{lipsum}
\usepackage{amsmath}
%\usepackage{apacite}
\usepackage[round]{natbib}

\begin{document}

%\def\mymacro#1{{\it #1}}

%\title{December 25, 2017}


\section*{\large December 28, 2017}

Few days are remaining from 2017 and I am racing with time to finish revision of papers due. I am also trying to run as many granular physics simulations (for studying friction) as I need and can on \href{https://www.princeton.edu/researchcomputing/index.xml}{Princeton Research Computing} Tiger cluster at this relatively quiet time with the cluster (most nodes are empty, no queue time). I also need to finish draft of a first paper on pattern (and cyclones!) formation on perturbed granular beds!

In the meantime and while simulations are running, I will be visiting the \href{http://whitney.org}{Whitney Museum of American Art} tomorrow with friends and colleagues in NYC! It is going to be a very cold day, but I cannot wait for that!
  


\section*{\large December 25, 2017}
\vspace{-0.1 in}

I just spent five hours of my Christmas Monday on installing {\bf gfortran} on MacOS High Sierra! I was getting the following error while running some simple Fortran codes for earthquake source simulations. 

\begin{ppl}
FATAL:/opt/local/bin/../libexec/as/x86-64/as: I don't understand 'm' flag! 
\end{ppl}

I thought it is due to some configuration problems of the latest version of gfortran --that I installed from the \href{https://gcc.gnu.org/wiki/GFortranBinariesMacOS}{GNU project}-- with recent MacOS. I was right, it turned out that the latest releases (ver>7) of gfortran (maybe only for this particular use) is not compatible with OS High Sierra anymore. A solution that I found after hours of struggle, was to install an older version of fortran {\bf gcc6.4} that with itself has {\bf gfortran6}. This could be installed via {\bf MacPorts}:

\begin{ppl}
sudo port install gcc6
\end{ppl}

and will have the following name in the binaries folder: ``gfortran-mp-6''.


%




%\bibliographystyle{plainnat}
%\small
%\bibliography{library_segregation}
\end{document} 
  